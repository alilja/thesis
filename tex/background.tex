% lit_review.text - Andrew F. Lilja's MS Thesis literature review

\chapter{Background}

\section{Expertise}

\subsection{Experts}

Operationalizing \emph{expertise} is challenging, because with very few exceptions like chess ratings, differences between novices or even between different experts must be based on judgements of performance. In field experiments, peer judgements are primarily used, with experts defined as individuals who are recognized within their field who performs at the highest level \citep{Kahneman2009}. Experts are generally better at perceiving cues and determining information about the scenario based on those cues \citep{Nee2006, Klein1999}, as well as determining what information is not there \citep{Klein1992}. Importantly, expert mental models develop to a point where new information can be easily integrated into existing knowledge and used to change strategies when needed \citep{Glaser1996}.

When these conditions are not present, substantial experience may develop with the incorrect knowledge, resulting in \emph{imperfect intuition} \citep{Kahneman2009}. These individuals have substantial experience but a flawed mental model resulting in regularly incorrect choices. False expertise manifests in individuals as ``compelling intuitions even when they lack true skill'' \citep{Kahneman2009} Correcting these flaws requires substantial revision to the mental model \citep{Klein2006, Chi2008} to fill in missing information and replacing incorrect knowledge with accurate experiences.

% this ¶ needs work; basically, connect expertise to mental model and the importance of having a good one
Correct mental models allow for skilled decision making, with choices having predictable effects on a situation. These decisions require the integration of many variables and attributes in order to determine the best course of action \citep{Klein2008}. Experts with superior mental models are able to immediately make correct decisions \citep{Klein1992} by information about the scenario at hand with previous experiences to generate a course of action \citep{Klein1999}. Mental models must contain enough information that the presented cues can trigger them. Without a correct mental model, this triggering cannot occur, and decision makers are unable to integrate and act on information successfully \citep{Lipshitz1997}.

\section{Conceptual Change}

Learning is rarely a simple process of simply adding additional, correct information to existing stores of knowledge. \citet{Klein2006} describe this as the ``storehouse metaphor,'' where critical information is missing and can simply be added to existing knowledge. Instead, learning typically requires replacing existing, incorrect knowledge with more correct information, especially as learners know increasingly more. The process of correcting errors in knowledge or cognitive skills is known as \emph{conceptual change} \citep{Chi2008}, and involves changes in knowledge on one of three grain sizes: false beliefs, mental models, and ontology \citep{Gadgil2012}.

Misconceptions can be caused by a combination of these three grains, and each has different correction methodologies \citep{Gadgil2012}. \emph{False beliefs} are typically simple, incorrect beliefs about a single idea that can be described in a single sentence that can usually be corrected through direct refutation. \emph{Ontological} errors involve miscategorizing information or attributing processes to an incorrect category \citep{Chi1994}. Finally, \emph{Mental models} are cognitive representations of how things happen \citep{Klein2006} and allow learners to make assumptions and inferences \citep{Gadgil2012}.

\subsection{Changing Mental Models}

According to \citet{Chi2008}, mental models can be incorrect in two different ways. Those that are missing information are in conflict with the correct mental model insofar as they are incomplete, whereas a mental model can be complete with incorrect information, resulting in a flawed mental model. Correcting the former problem merely involves introduction of accurate information, while the latter requires a mental model transformation, replacing the incorrect information with new, correct knowledge. Flawed mental models are usually internally consistent; that is, learners can use their models to justify incorrect knowledge and statements \citep{Vosniadou1994}.

One highly effective tool for correcting these flaws is self-explanation \citep{Chi1994}, which involves students reconciling their existing mental model with a correct one by eliciting explanations when presented with new information. This process targets misconceptions at a false belief level by making students aware of the inconsistencies between their model and the correct one. Once aware of these inaccuracies, the accumulation of refutations eventually results in a correction to the flawed mental model \citep{Chi2008}.The presence of false beliefs in a mental model not only prevents an accurate model from forming, but can impair the development of new, correct mental models or enhance existing mental models in related ontologies \citep{Jacobson2013}.

However, on its own, self-explanation may not be completely effective. Mental model transformation is more challenging than false belief revision, due to the tendency of learners to reject conflicting information and erect knowledge shields \citep{Klein2006}. A key step of mental model transformation is that of unlearning, where incorrect beliefs are disconfirmed and successfully eliminated from the model, allowing for correct information to be integrated. Additionally, without awareness of the flaws in their models, they are unable to generate any self-explanations \citep{Chi2002} or mental model transformation \citep{Chi2008}.

This unlearning process can be hastened and enhanced by first demonstrating where the mental model is flawed, then reducing a learner's confidence in it before then gradually introducing the new concept \citep{Klein2006}. This \emph{bridging} process provides a number of examples and alternatives between the existing mental model and the correct mental model \citep{Brown1989}. By asking students to justify and self-explain their model at each step, learners are more likely to accept changes to their own concepts \citep{Brown1989, Chi2008, Chi1994, Klein2006}. This model, where learners are presented with proof that their model is incorrect and then gradually guided to the appropriate solution is a key method to correcting mental models.

Experts frequently learn through deliberate practice, which can be seen as a special case of self-explanation

\subsection{Use of Simulators}

Traditional self-explanation approaches involve refutation texts, where learners read and explain each line of a text designed to challenge incorrect beliefs about a mental model \citep{Chi1994}. A virtual environment can fill the same role by presenting learners opportunities to exercise their mental models in series of decision tasks, eliciting self-explanation either during or after a scenario has been completed. A key benefit of training in virtual environments is the ability to alter environment and task variables to observe changes in outcome \citep{Klein2006}: A choice can be made and the exact cause and effect can be immediately observed. Virtual environments allow for the ramifications of different decisions to be played out and examined, which provides the opportunity for a trainee to reflect on the benefits and faults of their mental models. By observing and describing the effects of a decision in the simulation, learners generate explanations for why a course of action resulted in a given outcome.

Learners with incorrect mental models can observe the outcomes of their choices and compare them directly to the outcomes of a decision made with a correct mental model, eliciting self-explanation in an interactive context: not only are their incorrect beliefs refuted by the presentation of correct information, students can directly observe how changes in their mental model would play out in a real-world scenario and incorporate that information directly into their mental models \citep{Salzman1999}. The simulation provides a frame of reference, which contextualizes the information: instead of creating a situation in their head, learners can simply observe what is presented to them in its actual, real-world context \citep{Ericsson1993}.

Because learning and cognitive transformation is ultimately a sensemaking process \citep{Klein2006}, of critical importance to skill learning and expertise is the validity of the environment the skill is learned in \citep{Kahneman2009}. A highly valid environment is one where the outcomes of events are consistent, predictable, and occur within a short period of time — but not necessarily without uncertainty. A virtual environment allows for all of these conditions to be met, with the special case of time compression, allowing for outcomes that may occur over a long temporal period to be observed immediately after a choice is made. Additionally, virtual environments allow for the learning of cues: important cues, like the presence or lack of certain environmental factors or information, can be emphasized and distractors can be limited or eliminated.

The flexibility of interfaces in a virtual environment allow for gradual alternative introduction. Instead of being confronted with only one correct alternative, trainees can be shown increasingly correct decisions played out in the simulated environment. By presenting the information using a virtual environment and contextualized in a real-world scenario, trainees are more able to interact with the new information, explore its ramifications \citep{Ericsson1993, Patterson2010}, compare and contrast their choices with the correct option \citep{Fowlkes2009}, and easily integrate this information into the mental model that matches the simulated environment \citep{Salzman1999}.
