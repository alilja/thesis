% lit_review.tex - Andrew F. Lilja's MS Thesis literature review

\chapter{Background}

Learning, decision making, and expertise are defined by the mental models behind them. Mental models are cognitive constructs that contain causal information about how systems work. A strong model means better inferences and predictions about those systems, while misconceptions can lead to mistakes in judgments and impairments to learning. Repairing a flawed mental model involves identifying false beliefs and errors in knowledge and choosing the appropriate strategy to correct those flaws. One robust tool for accomplishing this is self-explanation, where learners are presented with information that conflicts with their existing model and asked to reconcile what they believe with the truth. Self-explanation can be elicited in a simulator and the gains made there can translate to real-world situations.

\section{Experts}

Determining an individual's level of expertise is challenging, because with very few exceptions (like chess ratings) differences between novices and experts --- or even between different experts --- must be based on judgments of performance, not quantifiable rankings. It is typical for peer assessments to be used in field experiments, where experts are defined as individuals recognized within their field who perform at the highest levels \citep{Kahneman2009}. Experts are generally better at perceiving situational cues and elucidating information from them \citep{Nee2006, Klein1999} as well as identifying what information is missing \citep{Klein1992}. Importantly, the mental models of experts are developed to a point where new information can be easily integrated into existing knowledge and used to change strategies when needed \citep{Glaser1996}.

Because learning and cognitive transformation is ultimately a process of sensemaking\sidenote[][-2em]{The process of taking stimuli, cues, and circumstances and synthesizing plausible meaning from them \citep{Weick2005}.} \citep{Klein2006}, a highly valid environment is of critical importance to skill learning and expertise \citep{Kahneman2009}. An environment is valid when the outcomes of events are consistent, predictable, and occur within a short period of time --- but not necessarily without uncertainty. The development of expertise also requires deliberate practice, goal-oriented behaviors that are the most effective at enhancing existing skills \citep{Ericsson2008}. Deliberate practice requires motivation on the part of the learner, instructional material based on the learner's existing skill level, and immediate, accurate feedback about the results of their actions \citep{Ericsson1993}.

When these conditions are not present, substantial experience may develop based on the incorrect knowledge, resulting in \emph{imperfect intuition} \citep{Kahneman2009}. These individuals have been engaged in their field for similar lengths of time as true experts, but their flawed mental models regularly result in incorrect choices. False expertise manifests in individuals as ``compelling intuitions even when they lack true skill'' \citep{Kahneman2009}: they wholly believe their judgments are correct when they are not. Correcting these flaws requires substantial revision to the mental model to fill in missing information and replace false beliefs with correct knowledge \citep{Klein2006, Chi2008}.

Correct and robust mental models allow for skilled decision making, with choices having predictable effects on a situation. Expert decisions require the integration of many environmental variables, decision attributes, and options in order to determine the best course of action \citep{Klein2008}. Superior mental models allow experts to immediately make correct decisions \citep{Klein1992} by combining information about the scenario at hand with previous experiences to generate a potential course of action \citep{Klein1999}. Thus, to be effective, mental models must contain enough information that the presented cues can trigger them. Without a correct mental model, this triggering cannot occur, and decision makers are unable to integrate and act on information successfully \citep{Lipshitz1997}.

\section{Novices}

Frequently, novices are defined in relation to experts --- lacking any special knowledge or skills or the ability to consistently make correct choices \citep{Lipshitz1997}. Functionally, when making decisions with a flawed mental model, the decision maker performs more like a novice than an expert. Experts tend to analyze the situation as a whole \citep{Calderwood1990}, integrating information to form a mental model about the situation \citep{Gott1986} while novices tend to simply evaluate their options, without considering other information \citep{Calderwood1990}. Novices lack well-developed mental models, preventing them from efficiently storing and tracking new knowledge. Without this ability, novices engage in repeated requests for the same information \citep{Lipshitz1997} and an inability to recognize patterns \citep{Shanteau1988} or derive meaning from those patterns \citep{Means1993}.

\section{Conceptual Change}

Learning is rarely the process of simply adding additional material to existing stores of knowledge. \citet{Klein2006} describe this concept as the ``storehouse metaphor,'' where critical information is missing and can simply be added to existing knowledge or used to fill in gaps. Instead, learning typically requires replacing existing, incorrect knowledge with correct information, especially in the case of learners with lots of experience in topic. The process of correcting errors in procedural knowledge and cognitive skills is known as \emph{conceptual change} \citep{Chi2008}, and involves changes in knowledge in one of three grain sizes: beliefs, ontology, and mental models \citep{Gadgil2012}.

Misconceptions at one level can be caused by a combination of grains at a lower level, and correcting one may require a grain-specific approach \citep{Gadgil2012}. \emph{False beliefs} are typically simple, incorrect ideas that can be described in a single sentence. Due to their simplicity, they can usually be corrected through direct refutation. \emph{Ontological} errors involve either mis-categorizing information or attributing processes to an incorrect category \citep{Chi1994}. Finally, \emph{mental models} are cognitive representations of how things happen \citep{Klein2006} and allow learners to make assumptions and inferences \citep{Gadgil2012}. Flaws in mental models are usually internally consistent and do not prevent assumptions and inferences from being made, but contain a fundamental inaccuracy that leads to mistaken predictions about the outcomes of the system \citep{Vosniadou1994}.

\section{Mental Model Transformation}

The presence of false beliefs in a mental model not only prevents an accurate model from forming and developing, but can impair the development of new, correct mental models or the enhancement of existing mental models in related ontologies \citep{Jacobson2013}. According to \citet{Chi2008}, mental models can be incorrect in two different ways. Models that are missing information are in conflict with the correct mental model insofar as they are incomplete, while a mental model can be complete but contain incorrect information. Correcting the former problem merely involves introduction of accurate information into gaps in the model, while the latter requires transformation of the mental model, replacing the incorrect information with new, correct knowledge.

One highly effective technique for correcting a flaws is self-explanation \citep{Chi1994}, which involves students comparing their existing mental model against a correct one and eliciting explanations when presented with the new information. This process targets misconceptions at a false belief level by making students aware of the inconsistencies between their model and forcing them to reconcile the differences. Once aware of these inaccuracies, the accumulation of refutations eventually results in a correction of the flawed mental model \citep{Chi2008}. This two-step process of identifying a flaw and providing a correct alternative is critical to successfully refuting false beliefs and integrating them on a model-wide scope \citep{Chi1994, Klein2006}.

On its own, however, self-explanation may not be completely effective. When learners are not aware of flaws in their mental models, they are unable to generate any self-explanations \citep{Chi2002} that lead mental model transformation \citep{Chi2008}. Transformation on the level of the mental model is more challenging than false belief revision due to the tendency of learners to reject conflicting information and erect knowledge shields,\sidenote[][0em]{Rationales used to deflect information that conflicts with an existing mental model. See \citet{Chinn1993} for a list of deflection reactions.} especially as their models become more complex and advanced. A key step of mental model transformation is that of \emph{unlearning}, where incorrect beliefs are dis-confirmed and successfully eliminated from the model, allowing for correct information to be integrated. Without unlearning, new information and feedback from the environment cannot be integrated, preventing the development of additional knowledge \citep{Klein2006}.

The conceptual change process can be hastened and made more robust by gradually introducing the new concept. \citep{Klein2006}. Immediately confronting the learner with conflicting information may result in rejection of that new knowledge. Instead, the gradual \emph{bridging} process provides a number of increasingly correct examples between the existing mental model and the correct mental model \citep{Brown1989}. Asking students to justify and self-explain at each step means that learners are more likely to accept changes to their models \citep{Brown1989, Chi2008, Chi1994, Klein2006}. This process, where learners are presented with proof that their model is incorrect and then gradually guided to the appropriate solution, is a key method to correcting mental models.

\section{Virtual Environments}

Traditional self-explanation approaches involve refutation texts, where learners read and explain each line of a text designed to challenge incorrect beliefs in a mental model about a process \citep{Chi1994}. A virtual environment can fill the same role by presenting learners opportunities to exercise their mental models in series of decision tasks, eliciting self-explanation either during or after a scenario has been completed. A key benefit of training in virtual environments is the ability to alter environment and task variables to observe changes in outcome \citep{Klein2006}: A choice can be made, and the exact cause and effect can be immediately observed. Virtual environments allow for a wide range of different decisions to be played out and examined in the same environmental context, which provides the opportunity for a trainee to reflect on the benefits and faults of their mental models \citep{Salzman1999}. By observing and describing the effects of a decision in the virtual environment, learners generate self-explanations for why a course of action resulted in a given outcome.

Learners with incorrect mental models can observe the results of their choices and compare them directly to the outcomes of a decision made from a correct mental model, eliciting self-explanation in an interactive context. Not only are their incorrect beliefs refuted by the presentation of correct information, students can directly observe how changes in their mental model would play out in a real-world scenario and incorporate that information directly into their mental models \citep{Salzman1999}. The virtual environment provides a frame of reference, which contextualizes the information: instead of running an imaginary simulation in their head, learners can simply observe what is presented to them in its actual, real-world context \citep{Erickson1993}.

A critical factor for the development of expertise and a strong mental model is the validity of the learning environment\sidenote[][0em]{See \S1.1.} \citep{Klein2006, Kahneman2009}. A virtual environment is highly consistent and can also compress time, allowing for outcomes that may occur over a long temporal period to be observed immediately after a choice is made. Additionally, because virtual environments can be used to directly manipulate the world around the learner, important cues can be emphasized and distractors can be limited or eliminated \citep{Salzman1999}. Distinguishing between valuable and unproductive information is a key skill for expert decision makers \citep{Klein1992}, and by subtly modifying the surrounding environment, a virtual environment can shape a learner's understanding of what environmental cues and factors to pay attention to.

The flexibility of interfaces in a virtual environment allow for gradual alternative introduction \citep{Erickson1993}. Instead of being confronted with only one correct alternative, trainees can be shown increasingly correct decisions and allowed to observe how they play out in the simulated environment. By presenting the information using a virtual environment and contextualized in a real-world scenario, trainees are more able to interact with the new information, explore its ramifications \citep{Erickson1993, Patterson2010}, compare and contrast their choices with the correct option \citep{Fowlkes2009}, and easily integrate this information into the mental model that matches the simulated environment \citep{Salzman1999}.

Skills acquired in a virtual environment readily carry over to non-simulated environments \citep{Cromby1996, Watanuki2007}, even in high-skill tasks like surgery \citep{Seymour2002} and firefighting \citep{Bliss1997}. This is due to the high ecological validity of virtual environments, which when combined with the experimental flexibility of complete control over the simulation provides a powerful tool for training and education \citep{Loomis1999}. A virtual training simulation can be used to guide the development of mental models, correct errors, and prevent flaws from by providing the environment required for the development of expertise and expert-level decision making.
