% lit_review.text - Andrew F. Lilja's MS Thesis literature review

\chapter{Background}

\section{Expertise}

\subsection{Experts}

Operationalizing expertise is challenging, because with very few exceptions (like chess ratings) differences between novices and experts --- or even between different experts ---must be based on judgements of performance. It is typical for peer assessments to used in field experiments, where experts are defined as individuals recognized within their field as performing at the highest level \citep{Kahneman2009}. Experts are generally better at perceiving situational cues and elucidating information from them \citep{Nee2006, Klein1999}, as well as identifying what information is missing \citep{Klein1992}. Importantly, the mental models of experts are developed to a point where new information can be easily integrated into existing knowledge and used to change strategies when needed \citep{Glaser1996}.

% ¶ on conditions for expertise
Because learning and cognitive transformation is ultimately a sensemaking process \citep{Klein2006}, a highly valid environment is of critical importance to skill learning and expertise \citep{Kahneman2009}. A highly valid environment is one where the outcomes of events are consistent, predictable, and occur within a short period of time — but not necessarily without uncertainty.

When these conditions are not present, substantial experience may develop with the incorrect knowledge, resulting in \emph{imperfect intuition} \citep{Kahneman2009}. These individuals have been engaged in their field for similar lengths of time as true experts, but their flawed mental models regularly result in incorrect choices. False expertise manifests in individuals as ``compelling intuitions even when they lack true skill'' \citep{Kahneman2009}: they wholly believe their judgements are correct when they are not. Correcting these flaws requires substantial revision to the mental model to fill in missing information and replace false beliefs with correct knowledge \citep{Klein2006, Chi2008}.

% this ¶ needs work; basically, connect expertise to mental model and the importance of having a good one
Correct and robust mental models allow for skilled decision making, with choices having predictable effects on a situation. Expert decisions require the integration of many environmental variables, decision attributes, and options in order to determine the best course of action \citep{Klein2008}. Superior mental models allow experts to immediately make correct decisions \citep{Klein1992} by combining information about the scenario at hand with previous experiences to generate a potential course of action \citep{Klein1999}. Thus, to be effective, mental models must contain enough information that the presented cues can trigger them. Without a correct mental model, this triggering cannot occur, and decision makers are unable to integrate and act on information successfully \citep{Lipshitz1997}.

\subsection{Novices}


\subsection{Expert and Novice Differences}


\section{Conceptual Change}

Learning is rarely the process of simply adding additional material to existing stores of knowledge. \citet{Klein2006} describe this concept as the ``storehouse metaphor,'' where critical information is missing and can simply be added to existing knowledge or used to fill in gaps. Instead, learning typically requires replacing existing, incorrect knowledge with correct information, especially in the case of learners with lots of experience in topic. The process of correcting errors in procedural knowledge and cognitive skills is known as \emph{conceptual change} \citep{Chi2008}, and involves changes in knowledge in one of three grain sizes: beliefs, mental models, and ontology \citep{Gadgil2012}.

Misconceptions at one level can be caused by a combination of grains at a lower level, and correcting one may require a grain-specific approach \citep{Gadgil2012}. \emph{False beliefs} are typically simple, incorrect ideas that can be described in a single sentence. Due to their simplicity, they can usually be corrected through direct refutation. \emph{Ontological} errors involve either miscategorizing information or attributing processes to an incorrect category \citep{Chi1994}. Finally, \emph{mental models} are cognitive representations of how things happen \citep{Klein2006} and allow learners to make assumptions and inferences \citep{Gadgil2012}. Flaws in mental models are usually internally consistent and do not prevent assumptions and inferences from being made, but contain a fundamental inaccuracy that leads to mistaken predictions about the outcomes of system \citep{Vosniadou1994}.

\subsection{Mental Model Transformation}

The presence of false beliefs in a mental model not only prevents an accurate model from forming, but can impair the development of new, correct mental models or enhance existing mental models in related ontologies \citep{Jacobson2013}. According to \citet{Chi2008}, mental models can be incorrect in two different ways. Models that are missing information are in conflict with the correct mental model insofar as they are incomplete, while a mental model can be complete but contain incorrect information. Correcting the former problem merely involves introduction of accurate information into gaps in the model, while the latter requires transformation of the mental model, replacing the incorrect information with new, correct knowledge.

One highly effective technique for correcting a flaws is self-explanation \citep{Chi1994}, which involves students comparing their existing mental model against a correct one and eliciting explanations when presented with new information. This process targets misconceptions at a false belief level by making students aware of the inconsistencies between their model and forcing them to reconcile the differences. Once aware of these inaccuracies, the accumulation of refutations eventually results in a correction to the flawed mental model \citep{Chi2008}. This two-step process of identifying a flaw and providing a correct alternative is critical to successfully refuting false beliefs and integrating them on a model-wide level \citep{Chi1994, Klein2006}.

On its own, however, self-explanation may not be completely effective. Mental model transformation is more challenging than false belief revision, due to the tendency of learners to reject conflicting information and erect knowledge shields \citep{Klein2006}. A key step of mental model transformation is that of \emph{unlearning}, where incorrect beliefs are disconfirmed and successfully eliminated from the model, allowing for correct information to be integrated. Additionally, when learners are not aware of flaws in their mental models, they are unable to generate any self-explanations \citep{Chi2002} that leads mental model transformation \citep{Chi2008}.

The transformation process can be hastened and made more robust by first demonstrating where the mental model is flawed followed by reducing a learner's confidence in it before finally introducing the new concept in gradual steps \citep{Klein2006}. To avoid abruptly denying the learner's beliefs, this \emph{bridging} process provides a number of examples between the existing mental model and the correct mental model \citep{Brown1989}. By asking students to justify and self-explain their model at each step, learners are more likely to accept changes to their own concepts \citep{Brown1989, Chi2008, Chi1994, Klein2006}. This process, where learners are presented with proof that their model is incorrect and then gradually guided to the appropriate solution, is a key method to correcting mental models.

% ¶ needs to exist
Experts frequently learn through deliberate practice, which can be seen as a special case of self-explanation

\subsection{Virtual Environments}

Traditional self-explanation approaches involve refutation texts, where learners read and explain each line of a text designed to challenge incorrect beliefs in a mental model about a process \citep{Chi1994}. A virtual environment can fill the same role by presenting learners opportunities to exercise their mental models in series of decision tasks, eliciting self-explanation either during or after a scenario has been completed. A key benefit of training in virtual environments is the ability to alter environment and task variables to observe changes in outcome \citep{Klein2006}: A choice can be made, and the exact cause and effect can be immediately observed. Virtual environments allow for a wide range of different decisions to be played out and examined in the same environmental context, which provides the opportunity for a trainee to reflect on the benefits and faults of their mental models \citep{Salzman1999}. By observing and describing the effects of a decision in the virtual environment, learners generate self-explanations for why a course of action resulted in a given outcome.

Learners with incorrect mental models can observe the results of their choices and compare them directly to the outcomes of a decision made from a correct mental model, eliciting self-explanation in an interactive context. Not only are their incorrect beliefs refuted by the presentation of correct information, students can directly observe how changes in their mental model would play out in a real-world scenario and incorporate that information directly into their mental models \citep{Salzman1999}. The virtual environment provides a frame of reference, which contextualizes the information: instead of running an imaginary simulation in their head, learners can simply observe what is presented to them in its actual, real-world context \citep{Ericsson1993}.

As previously discussed, a critical factor for the development of expertise and a strong mental model is the validity of the learning environment \citep{Klein2006, Kahneman2009}. A virtual environment is not only highly consistent, it can also compress time, allowing for outcomes that may occur over a long temporal period to be observed immediately after a choice is made. Additionally, because virtual environments can be used to directly manipulate the world around the learner, important cues like the visibility of certain environmental factors or information, can be emphasized and distractors can be limited or eliminated \citep{Salzman1999}. Due to the high environmental validity of virtual environments \citep{Loomis1999}, skills developed in the virtual environment readily transfer to non-simulated environments \citep{Cromby1996, Watanuki2007}, including high-skill tasks like surgery \citep{Seymour2002} and firefighting \citep{Bliss1997}.

The flexibility of interfaces in a virtual environment allow for gradual alternative introduction \citep{Ericsson1993}. Instead of being confronted with only one correct alternative, trainees can be shown increasingly correct decisions and allowed to observed how they play out in the simulated environment. By presenting the information using a virtual environment and contextualized in a real-world scenario, trainees are more able to interact with the new information, explore its ramifications \citep{Ericsson1993, Patterson2010}, compare and contrast their choices with the correct option \citep{Fowlkes2009}, and easily integrate this information into the mental model that matches the simulated environment \citep{Salzman1999}.
