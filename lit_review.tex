% lit_review.text - Andrew F. Lilja's MS Thesis literature review

\documentclass{book}
\begin{document}

\chapter{Background}

\section{Expertise}


\section{Conceptual Change}

Learning is rarely a simple process of simply adding additional, correct information to existing stores of knowledge. Klein and Baxter (2006) describe this as the ``storehouse metaphor,'' where critical information is missing and can simply be taught. Instead, learning typically requires replacing existing, incorrect knowledge with more correct information, especially as learners know increasingly more. The process of correcting errors in knowledge or cognitive skills is known as \emph{conceptual change} (Chi 2008), and involves changes in knowledge on one of three grain sizes: false beliefs, mental models, and ontology (Gadgil, Nokes-Malach, & Chi 2012).

Misconceptions can be caused by a combination of these three grains, and each has different correction methodologies (Gadgil et al. 2012). \emph{False beliefs} are typically simple, incorrect beliefs about a single idea that can be described in a single sentence that can usually be corrected through direct refutation. \emph{Ontological} errors involve miscategorizing information or attributing processes to an incorrect category (Chi, de Leeuw, Chiu, & LaVancher 1994). Finally, \emph{Mental models} are cognitive representations of how things happen (Klein and Baxter 2006) and allow learners to make assumptions and inferences (Gadgil et al. 2012).

\subsection{Changing Mental Models}

According to Chi (2008), mental models can be incorrect in two different ways. Those that are missing information are in conflict with the correct mental model insofar as they are incomplete, whereas a mental model can be complete with incorrect information, resulting in a flawed mental model. Correcting the former problem merely involves introduction of accurate information, while the latter requires a mental model transformation, replacing the incorrect information with new, correct knowledge. Flawed mental models are usually internally consistent; that is, learners can use their models to justify incorrect knowledge and statements (Vosniadou 1994).

One highly effective tool for correcting these flaws is self-explanation (Chi, de Leeuw, Chiu, & LaVancher 1994), which involves students reconciling their existing mental model with a correct one by eliciting explanations when presented with new information. This process targets misconceptions at a false belief level by making students aware of the inconsistencies between their model and the correct one. Once aware of these inaccuracies, the accumulation of refutations eventually results in a correction to the flawed mental model (Chi 2008).

However, on its own, self-explanation may not be completely effective. Mental model transformation is more challenging than false belief revision, due to the tendency of learners to reject conflicting information and erect knowledge shields (Klein & Baxter 2006). A key step of mental model transformation is that of unlearning, where incorrect beliefs are disconfirmed and successfully eliminated from the model, allowing for correct information to be integrated. Additionally, without awareness of the flaws in their models, they are unable to generate any self-explanations (Chi 2008).

\subsection{Use of Simulators}

Traditional self-explanation approaches involve refutation texts, where learners read and explain each line of a text designed to challenge incorrect beliefs about a mental model (Chi, de Leeuw, Chiu, & LaVancher 1994). A simulator can fill the same role by presenting learners opportunities to exercise their mental models in series of decision tasks, eliciting self-explanation either during or after a scenario has been completed. A key benefit of training in simulators is the ability to alter environment and task variables to observe changes in outcome (Klein & Baxter 2009). Simulators allow for the ramifications of different decisions to be played out and examined, which provides the opportunity for a trainee to reflect on the benefits and faults of their mental models. By observing and describing the effects of a decision in the simulation, learners generate explanations for why a course of action resulted in a given outcome.

Learners with incorrect mental models can observe the outcomes of their choices and compare them directly to the outcomes of a decision made with a correct mental model, eliciting self-explanation in an interactive context: not only are their incorrect beliefs refuted by the presentation of correct information, students can directly observe how changes in their mental model would play out in a real-world scenario and incorporate that information directly into their mental models (Salzman, Dede, Loftin, & Chen 1999). The simulation provides a frame of reference, which contextualizes the information: instead of creating a situation in their head, learners can simply observe what is presented to them in its actual, real-world context (Erickson 1993).

Because learning and cognitive transformation is ultimately a sensemaking process (Klein & Baxter 2009), of critical importance to skill learning and expertise is the validity of the environment the skill is learned in (Kahneman & Klein 2009). A highly valid environment is one where the outcomes of events are consistent, predictable, and occur within a short period of time — but not necessarily without uncertainty. A simulator allows for all of these conditions to be met, with the special case of time compression, allowing for outcomes that may occur over a long temporal period to be observed immediately after a choice is made. Additionally, simulators allow for the learning of cues: important cues, like the presence or lack of certain environmental factors or information, can be emphasized and distractors can be limited or eliminated.

